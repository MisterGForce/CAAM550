\documentclass{article} % \documentclass{} is the first command in any LaTeX code.  It is used to define what kind of document you are creating such as an article or a book, and begins the document preamble

\usepackage{amsmath} % \usepackage is a command that allows you to add functionality to your LaTeX code
\usepackage{amsfonts}
\usepackage{amssymb}
\usepackage{siunitx}
\setlength{\parindent}{0pt}

\begin{document} % All begin commands must be paired with an end command somewhere
\textbf{Michael Goforth} \\
\textbf{CAAM 550} \\
\textbf{HW 12} \\
\textbf{12/03/2021} \\ 

\textbf{Problem 1} \\
\textbf{part a.}
\begin{align*}
&y_{k+2} = 3y_k - 2y_{k+1} +h(f(x_k, y_k) + 3f(x_{k+1}, y_{k+1})) \\
&-3y_k + 2y_{k+1} + y_{k+2} = h(f(x_k, y_k) + 3f(x_{k+1}, y_{k+1})) \\
&\alpha_0 = -3, \: \alpha_1 = 2, \: \alpha_2 = 1 \\
&\beta_0 = 1, \: \beta_1 = 3, \: \beta_2 = 0
\end{align*}
\textbf{Zero-stability}:
\begin{align*}
-3 + 2\gamma + \gamma^2 = 0 \\
(\gamma + 3)(\gamma - 1) = 0 \\
\gamma_1 = -3, \: \gamma_2 = 1
\end{align*}
The root $\gamma_1$ is not inside the unit disc so this multi-step method is not zero stable.\\
\textbf{Consistent}:
\begin{align*}
\sum_{j=0}^{2}\alpha_j &= -3 + 2 + 1 = 0 \\
\sum_{j=0}^{2}(\alpha_j  j - \beta_j) &= (-3(0)-1) + (2(1)-3)+(1(2)-0)=0
\end{align*}
So multi-step method is consistent. \\
\textbf{Truncation Error}
\begin{align*}
\sum_{j=0}^{2} \frac{j^2}{2}\alpha_j - \sum_{j=0}^{2}j \beta_j = 0(-3) + \frac{1}{2}(2) + 2(1) - 0(1) - 1(3) - 2(0) = 0 \\
\sum_{j=0}^{2} \frac{j^3}{6}\alpha_j - \sum_{j=0}^{2}\frac{j^2}{2} \beta_j = 0(-3) + \frac{1}{6}(2) + \frac{4}{3}(1) - 0(1) - \frac{1}{2}(3) - 2(0) = \frac{3}{2} \neq 0
\end{align*}
So the local truncation error is order($h^2$). \\
\\
\textbf{part b.} 
\begin{align*}
&y_{k+2} = \frac{1}{2}(y_k + y_{k+1}) + 2hf(x_{k+1}, y_{k+1}) \\
&-\frac{1}{2} y_k -\frac{1}{2} y_{k+1} + y_{k+2} = 2hf(x_{k+1}, y_{k+1}) \\
&\alpha_0 = -\frac{1}{2}, \: \alpha_1 = -\frac{1}{2}, \: \alpha_2 = 1 \\
&\beta_0 = 0, \: \beta_1 = 2, \: \beta_2 = 0
\end{align*}
\textbf{Zero-stability}:
\begin{align*}
-\frac{1}{2} - \frac{1}{2} \gamma + \gamma^2 = 0 \\
(\gamma - 1)(\gamma + \frac{1}{2}) = 0 \\
\gamma_1 = 1, \: \gamma_2 = -\frac{1}{2}
\end{align*}
$\gamma_2$ is in the unit disk.  $\gamma_1$ is on the edge of the unit disk but is a simple root so this multi-step method is zero-stable. \\
\textbf{Consistent}: 
\begin{align*}
\sum_{j=0}^{2}\alpha_j &= -\frac{1}{2} -\frac{1}{2} + 1 = 0 \\
\sum_{j=0}^{2}(\alpha_j  j - \beta_j) &= (-\frac{1}{2}(0)-0) + (-\frac{1}{2}(1)-2)+(1(2)-0) = -\frac{1}{2}\neq 0
\end{align*}
So this multi-step method is not consistent. \\
\\


\textbf{Problem 2} \\
\textbf{part a} \\
See Jupyter notebook for code and results. \\

\textbf{part b} \\
See Jupyter notebook for code and results. \\

\textbf{part c} \\
\textbf{part i.} \\
See Jupyter notebook for code and results. \\
AB4 is accurate on the interval [-1, -.5], forward Euler is accurate over the whole interval [-1, 3]. \\
\textbf{part ii.} \\
See Jupyter notebook for code and results. \\
This stiff ODE is very sensitive to changes in the $h$ value. \\
\textbf{part iii.} \\
See Jupyter notebook for code and results. \\
\textbf{part iv.} \\
See Jupyter notebook for code and results. \\
These results are not surprising.  Truncation error for forward Euler method is O($h$), AB2 is O($h^2$), AB4 is O($h^3$), and RK4 is O($h^4$) although RK4 error was computed using a larger $h$ value than the rest. \\

\textbf{part d} \\
See Jupyter notebook for code and results. \\
This multistep method is not zero stable, so small errors will eventually grow unbounded.
\\


\textbf{Problem 3} \\

\textbf{part a} \\
\begin{align*}
\hat{y} = y_k + \frac{h}{4}(f(x_k, y_k) + f(\hat{x}, \hat{y})) \\
y_{k+1} = \frac{1}{3}(4 \hat{y} - y_k + hf(x_{k+1}, y_{k+1})) \\
f(x, y(x)) = y'(x) = \lambda y(x)
\end{align*}
Substituting in the right-hand size of the ODE into the first equation gives
\begin{align*}
\hat{y} = y_k + \frac{h}{4}(\lambda y_k + \lambda\hat{y}) \\
\hat{y} - \frac{h}{4} \lambda\hat{y} = y_k + \frac{h}{4}\lambda y_k \\
\hat{y} = \frac{1 + \frac{h\lambda}{4}}{1 - \frac{h\lambda}{4}} y_k \\
\hat{y} = \frac{4 + h\lambda}{4 - h\lambda} y_k \\
\end{align*}
Substituting this result and the right hand side into the equation for $y_{k+1}$ yields
\begin{align*}
y_{k+1} = \frac{1}{3}(4 (\frac{4 + h\lambda}{4 - h\lambda} y_k) - y_k + h\lambda y_{k+1}) \\
y_{k+1} - \frac{h\lambda}{3} y_{k+1} = \frac{4(4 + h\lambda)}{3(4 - h\lambda)} y_k - \frac{1}{3} y_k \\
(1-\frac{h\lambda}{3})y_{k+1} = (\frac{4(4 + h\lambda)}{3(4 - h\lambda)} - \frac{1}{3})y_k \\
y_{k+1} = (\frac{4(4 + h\lambda)}{(4 - h\lambda)(3 - h\lambda)} - \frac{1}{3 - h\lambda})y_k
\end{align*}
\\

\textbf{part b} \\
This method is stable if $|g(h\lambda)| < 1$, resulting in
\begin{equation}
\left|\frac{4(4 + h\lambda)}{(4 - h\lambda)(3 - h\lambda)} - \frac{1}{3 - h\lambda}\right| < 1
\end{equation}
See Jupyter notebook for code and results.  
\\


\textbf{Problem 4} 
\begin{align*}
\hat{y} &= y_k + \frac{h}{4}(f(x_k, y_k) + f(\hat{x}, \hat{y}))
\end{align*}
Using Taylor expansion:
\begin{align*}
f(\hat{x}, \hat{y}) &= f(x_k + h/2, y(x_k + h/2) = y'(x_k) + \frac{1}{2}hy''(x_k) + \frac{1}{8}hy'''(x_k) + \hdots \\
\hat{y} &= y_k + \frac{h}{4}(y'(x_k) + y'(x_k) + \frac{1}{2}hy''(x_k) + \frac{1}{8}h^2 y'''(x_k) + \hdots) \\
\hat{y} &= y_k + \frac{h}{2}y'(x_k) + \frac{1}{8}h^2 y''(x_k) + \frac{1}{32}h^3y'''(x_k) + \hdots
\end{align*}
Then
\begin{align*}
y_{k+1} = \frac{1}{3}(4 \hat{y} - y_k + hf(x_{k+1}, y_{k+1}))
\end{align*}
Using Taylor expansion once more
\begin{align*}
f(x_{k+1}, y_{k+1}) &= f(x_k+h, y(x_k+h)) = y'(x_k) + hy''(x_k) + \frac{1}{2}h^2y'''(x_k) + \hdots \\
y_{k+1}^* &= y(x_k+h) = y(x_k) +hy'(x_k) + \frac{1}{2}h^2y''(x_k)+ \frac{1}{6}h^3 y'''(x_k) + \hdots \\
y_{k+1} &= \frac{4}{3}\hat{y} - \frac{1}{3}y_k + \frac{1}{3}hf(x_{k+1}, y_{k+1}) \\
T_k &= \frac{y_{k+1}^* - y_{k+1}}{h}
\end{align*}
Order $h^{-1}$ terms (in numerator these terms are $h^1$):
\begin{align*}
h^{-1}(y(x_k) - \frac{4}{3}(y_k) + \frac{1}{3}y_k) = 0
\end{align*}
Order $h^0$ terms:
\begin{align*}
y'(x_k) - (\frac{4}{3}(\frac{1}{2}y'(x_k)) + \frac{1}{3}y'(x_k)) = 0
\end{align*}
So TR-BDF2 method is consistent. \\
Order $h^1$ terms:
\begin{align*}
h(\frac{1}{2}y''(x_k) - (\frac{4}{3}(\frac{1}{8}y''(x_k)) +\frac{1}{3}(y''(x_k))=0
\end{align*}
Order $h^2$ terms:
\begin{align*}
h^2(\frac{1}{6}y'''(x_k) - (\frac{4}{3}(\frac{1}{32}y'''(x_k)) +\frac{1}{3}(\frac{1}{2}y'''(x_k)) \neq 0
\end{align*}
Therefore the TR-BDF2 method is accurate to the second order.
\\





\end{document} % This is the end of the document


