\documentclass{article} % \documentclass{} is the first command in any LaTeX code.  It is used to define what kind of document you are creating such as an article or a book, and begins the document preamble

\usepackage{amsmath} % \usepackage is a command that allows you to add functionality to your LaTeX code
\usepackage{amsfonts}
\usepackage{amssymb}
\usepackage{siunitx}
\setlength{\parindent}{0pt}

\begin{document} % All begin commands must be paired with an end command somewhere
\textbf{Michael Goforth} \\
\textbf{CAAM 550} \\
\textbf{HW 10} \\
\textbf{11/25/2021} \\ 

\textbf{Problem 1} \\
See Jupyter notebook for code and results.
\\


\textbf{Problem 2} \\

\textbf{part i} \\
\begin{equation*}
\left|\int_a^b f(x)dx -\hat{T}(h)\right| \leq \left|\int_a^b f(x)dx -T(h) + T(h) - \hat{T}(h)\right|
\end{equation*}
\begin{align*}
&T(h) = \frac{h}{2}(f(a) + f(b)) + h \sum_{i=1}^{n-1}f(a+ih)  \\
&\hat{T}(h) = \frac{h}{2}(f(a) + \delta(a) + f(b)+ \delta(b)) + h \sum_{i=1}^{n-1}(f(a+ih)+ \delta(a+ih)) \\
&T(h) - \hat{T}(h) = \frac{h}{2}(\delta(a) + \delta(b)) + h \sum_{i=1}^{n-1}\delta(a+ih) \\
&T(h) - \hat{T}(h) \leq \frac{h}{2}(\delta + \delta) + h \sum_{i=1}^{n-1}\delta \\
&T(h) - \hat{T}(h) \leq (b-a)\delta \\
&\left|\int_a^b f(x)dx -T(h)\right| \leq \frac{b-a}{12}h^2 \max_{[a,b]}|f''(x)| \\
&\left|\int_a^b f(x)dx -\hat{T}(h)\right| \leq \frac{b-a}{12}h^2 \max_{[a,b]}|f''(x)| + (b-a)\delta
\end{align*}
\\

\textbf{part b} \\
\begin{equation*}
T(h) - \hat{T}(h) = \frac{h}{2}(\delta(a) + \delta(b)) + h \sum_{i=1}^{n-1}\delta(a+ih)
\end{equation*}
Because the $\delta$ function is random, on average this will equal $0$. However for small sample sizes (N), there could be significant variance here.  The standard error of the mean is $\sigma/\sqrt{n}$, so to reduce the effects of the variance a value of $n$ should be chosen so that it is large enough to reduce the variance.  For example, if the goal is less than 1\% error 95\% (2 sigma) of the time, then $\sigma/\sqrt{n} = .005$.  If sigma = 1 (as it does in the matlab randn function), then $n$ would need to be  40,000. \\
See Jupyter notebook for code and results.
\\


\textbf{Problem 3} \\

\textbf{part i} \\
See Jupyter notebook for code and results.  
\\

\textbf{part ii} \\
See Jupyter notebook for code and results.  
\\


\textbf{Problem 4} \\
\textbf{part i} \\
\begin{align*}
T_{0,0} &= T(h) = \frac{h}{2}f(a) + \frac{h}{2}f(b) \\
T_{1,0} &= T(h/2) = \frac{h}{4}f(a) +\frac{h}{2}f(a+\frac{h}{2}) + \frac{h}{4}f(b) \\
T_{1,1} &= T_{1,0} + \frac{1}{3}(T_{1,0} - T_{0,0}) \\
T_{1,1} &= \frac{h}{4}f(a) +\frac{h}{2}f(a+\frac{h}{2}) + \frac{h}{4}f(b) + \frac{h}{12}f(a) +\frac{h}{6}f(a+\frac{h}{2}) + \frac{h}{12}f(b) - \frac{h}{6}f(a) - \frac{h}{6}f(b) \\
T_{1,1} &= \frac{1}{6}f(a) + \frac{2}{3}f(a+\frac{h}{2}) + \frac{1}{6}f(b)
\end{align*}  
Also known as Simpson's rule.
\\





\end{document} % This is the end of the document


