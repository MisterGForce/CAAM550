\documentclass{article} % \documentclass{} is the first command in any LaTeX code.  It is used to define what kind of document you are creating such as an article or a book, and begins the document preamble

\usepackage{amsmath} % \usepackage is a command that allows you to add functionality to your LaTeX code
\setlength{\parindent}{0pt}

\begin{document} % All begin commands must be paired with an end command somewhere
\textbf{Michael Goforth} \\
\textbf{CAAM 550} \\
\textbf{HW 2} \\
\textbf{9/8/2021} \\ 

\textbf{Problem 2} \\
\textbf{part i.} \\
\begin{equation}
g(x) = x - \frac{f(x)}{d}
\end{equation}
\\
\textbf{part ii.} \\
Let $x_*$ be a fixed point of $g(x)$.  Then from theorem 0.1, we can say that if $|g'(x)|<1$, the fixed point iteration $x_{k+1}=g(x_k)$ converges.  So
\begin{equation}
|g'(x_*)|=|1 - \frac{f'(x_*)}{d}|<1
\end{equation}
Removing the absolute value leads us to the following 2 equations, that must both be true:

\begin{equation}
1 - \frac{f'(x_*)}{d}<1 \textrm{ and } 1 - \frac{f'(x_*)}{d}>-1
\end{equation}
Then,
\begin{equation}
- \frac{f'(x_*)}{d}<0 \textrm{ and } - \frac{f'(x_*)}{d}>-2
\end{equation}
\begin{equation}
f'(x_*)>0 \textrm{ and } f'(x_*)<2d
\end{equation}
Combining these gives us the condition
\begin{equation}
0 < f'(x_*) < 2d
\end{equation}
which if met will guarantee that this fixed point iteration is locally convergent.
\\
\\
\textbf{part iii.}
From theorem 0.1, the sequence converges q-linearly to $x_*$ with 
\begin{equation}
\lim_{k\to\infty} \frac{|x_{k+1}-x_*|}{|x_k-x_*|} = |g'(x_*)|=|1 - \frac{f'(x_*)}{d}|<1
\end{equation}
(See part iv. for special case where iteration converges quadratically.)
\\
\\
\textbf{part iv.}
From theorem 0.2, since $f$ is twice continuously differentiable, if 
\begin{equation}
g'(x_*)=0
\end{equation}
then the fixed point iteration converges to $x_*$ with q-order 2 (quadratic convergence).  So
\begin{equation}
g'(x_*)=1 - \frac{f'(x_*)}{d}=0
\end{equation}
\begin{equation}
\frac{f'(x_*)}{d}=1
\end{equation}
and finally
\begin{equation}
f'(x_*)=d
\end{equation}
\\
\\
\textbf{Problem 4.} \\
\textbf{part i.} \\
Find the derivative of function $\phi(\lambda)$ defined as:
\begin{equation}
\phi(\lambda) = \frac{1}{2}||Kf(\lambda)-g||_2^2-\frac{1}{2}||g-g^{true}||_2^2=0
\end{equation}
where $f(\lambda)$ is defined by the equation
\begin{equation}
(K^TK+\lambda I)f(\lambda)=K^Tg
\end{equation}
Let $\phi(\lambda)$ be considered a composition of the functions $F(X)$, $G(Y)$, and $f(\lambda)$, such that 
 \begin{equation}
\phi(\lambda) = F(G(f(\lambda)))
\end{equation}
and 
\begin{equation}
F(X) = \frac{1}{2}x^2+c
\end{equation}
\begin{equation}
G(Y) = K y - g
\end{equation}
Then using the matrix chain rule,
\begin{equation}
J_{\phi} = J_F J_G J_f
\end{equation}
\begin{equation}
J_F = X^T = (K f(\lambda) - g)^T
\end{equation}
\begin{equation}
J_G = K
\end{equation}
Using the implicit function theorem, we can define $f'(\lambda)$.  Let 
\begin{equation}
g(\lambda, f(\lambda)) = (K^TK+\lambda I)f(\lambda)=K^Tg
\end{equation}
Then by the implicit function theorem,
\begin{equation}
\frac{\partial g}{\partial\lambda} + \frac{\partial g}{\partial f(\lambda)}\frac{d f(\lambda)}{d \lambda} = 0
\end{equation}
\begin{equation}
I f(\lambda) + (K^T K + \lambda I)\frac{d f(\lambda)}{d \lambda} = 0
\end{equation}
and 
\begin{equation}
J_f =\frac{d f(\lambda)}{d \lambda} =  -(K^T K + \lambda I)^{-1} f(\lambda)
\end{equation}
Finally 
\begin{equation}
\phi'(\lambda) = J_{\phi}(\lambda) = -(K f(\lambda) - g)^T K (K^T K + \lambda I)^{-1} f(\lambda)
\end{equation}


\end{document} % This is the end of the document


