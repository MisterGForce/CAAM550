\documentclass{article} % \documentclass{} is the first command in any LaTeX code.  It is used to define what kind of document you are creating such as an article or a book, and begins the document preamble

\usepackage{amsmath} % \usepackage is a command that allows you to add functionality to your LaTeX code
\usepackage{amsfonts}
\usepackage{amssymb}
\usepackage{siunitx}
\setlength{\parindent}{0pt}

\begin{document} % All begin commands must be paired with an end command somewhere
\textbf{Michael Goforth} \\
\textbf{CAAM 550} \\
\textbf{HW 11} \\
\textbf{11/19/2021} \\ 

\textbf{Problem 1} \\
\textbf{part i.} \\
See Jupyter notebook for code and results. 
\\

\textbf{part ii.} \\
See Jupyter notebook for code and results. \\
\\


\textbf{Problem 2} \\
\textbf{part a} \\
Heun's method states
\begin{equation*}
y_{k+1} = y_k + \frac{h}{2}(f(x_k, y_k) + f(x_{k+1}, y_k+hf(x_k, y_k)))
\end{equation*}
Since $f(x, y(x)) = \lambda y(x)$, 
\begin{align*}
f(x_k, y_k) &= \lambda y_k \\
f(x_{k+1}, y_k+hf(x_k, y_k)) &= f(x_{k+1}, y_k + h\lambda y_k)) \\
f(x_{k+1}, y_k+hf(x_k, y_k)) &= \lambda (y_k + h\lambda y_k) \\
y_{k+1} &= y_k + \frac{h}{2}(\lambda y_k + \lambda (y_k + h\lambda y_k)) \\
y_{k+1} &= y_k + \frac{h\lambda}{2}y_k + \frac{h\lambda}{2}y_k + \frac{h^2\lambda^2}{2}\lambda y_k \\
y_{k+1} &= y_k(1 + h\lambda + \frac{h^2\lambda^2}{2}\lambda)
\end{align*}
\\

\textbf{part b} \\
The fourth order Runge-Kutta method is
\begin{align*}
Y_1 &= y_k \\
Y_2 &= y_k + \frac{h}{2}f(x_k, Y_1) \\
Y_3 &= y_k + \frac{h}{2}f(x_i + \frac{h}{2}, Y_2) \\
Y_4 &= y_k + hf(x_k+\frac{h}{2}, Y_3) \\
y_{k+1} &= y_k + \frac{h}{6}\left(f(x_k, Y_1) + 2f(x_i+\frac{h}{2}, Y_2), +2f(x_k+\frac{h}{2}, Y_3)+f(x_{k+1}, Y_4)\right)
\end{align*}
Using the fact that $f(x, y(x)) = \lambda y(x)$, 
\begin{align*}
Y_1 &= y_k \\
Y_2 &= y_k + \frac{h}{2}\lambda y_k \\
Y_3 &= y_k + \frac{h}{2}f(x_i + \frac{h}{2}, y_k + \frac{h}{2}\lambda y_k) \\
Y_3 &= y_k + \frac{h}{2}\lambda y_k + \frac{h^2 \lambda^2}{4} y_k \\
Y_4 &= y_k + hf(x_k+\frac{h}{2}, y_k + \frac{h}{2}\lambda y_k + \frac{h^2 \lambda^2}{4} y_k) \\
Y_4 &= y_k + h\lambda y_k + \frac{h^2 \lambda^2}{2}y_k + \frac{h^3 \lambda^3}{4}y_k \\
y_{k+1} &= y_k + \frac{h}{6}\left(\lambda Y_1 + 2\lambda Y_2 + 2\lambda Y_3 + \lambda Y_4\right) \\
y_{k+1} &= y_k \left(h\lambda + \frac{1}{2}h^2 \lambda^2 + \frac{1}{6}h^3 \lambda^3 + \frac{1}{24}h^4 \lambda^4 \right)
\end{align*}
\\

\textbf{part c} \\
Taylor series of the point $x_{k+1}$ expanded around $x_k$:
\begin{equation*}
y(x_{k+1}) = y(x_k+h) = y(x_k) + h y'(x_k) + \frac{h^2}{2}y''(x_k) + \frac{h^3}{6}y'''(x_k) +  \frac{h^4}{24}y''''(x_k) + \hdots
\end{equation*}
For the case $y'(x, y(x))=\lambda y(x)$, 
\begin{equation*}
y(x_{k+1}) = y(x_k+h) = y(x_k) + h \lambda y(x_k) + \frac{\lambda^2 h^2}{2}y(x_k) + \frac{\lambda^3 h^3}{6}y(x_k) +  \frac{\lambda^4 h^4}{24}y(x_k) + \hdots
\end{equation*}
Thus the Heun's method results in the second order Taylor series of $x_{k+1}$ expanded around $x_k$, and the fourth order Runge-Kutta method results in the fourth order Taylor series around the same point. \\

\textbf{part d} \\
See Jupyter notebook for code and results. \\
\\


\textbf{Problem 3} \\

\textbf{part a} \\
\begin{align*}
\frac{dy(x)}{dx} &= 4x - 2y(x) \\
x &\in [0,1] \\
y(0) &= 0
\end{align*}
Using the integrating factor $\phi(x) = e^{2x}$,
\begin{align*}
e^{2x}\frac{dy(x)}{dx} + 2e^{2x}y(x) &= e^{2x}4x \\
\frac{d}{dx}(e^{2x} y(x)) = e^{2x}4x \\
e^{2x} y(x) = \int e^{2x}4x dx \\
e^{2x} y(x) = 2x\,e^{2x} -e^{2x} + C\\
0 = y(0) = -1 + C \\
C = 1 \\
e^{2x} y(x) = 2x\,e^{2x} -e^{2x} + 1\\
y(x) = 2x - 1 + e^{-2x} \\
\end{align*}
\\

\textbf{part b} \\
See Jupyter notebook for code and results.  
\\

\textbf{part c} \\
See Jupyter notebook for code and results.  
\\

\textbf{part d} \\
See Jupyter notebook for code and results.  
\\


\textbf{Problem 4} \\
\begin{equation*}
T_k = \frac{y(x_{k+1})-y(x_k)}{h} - \frac{1}{2}(f(x_k, y(x_k)) + y(x_{k+1}, y_{k+1})) 
\end{equation*}
Using Taylor series expansion,
\begin{align*}
y(x_{k+1}) &= y(x_k+h) = y(x_k) + h y'(x_k) + \frac{h^2}{2}y''(x_k) + \frac{h^3}{6}y'''(\zeta_1) \\
y(x_{k+1}) -y(x_k) &=  h y'(x_k) + \frac{h^2}{2}y''(x_k) + \frac{h^3}{6}y'''(\zeta_1)
\end{align*}
and also using Taylor expansion
\begin{align*}
f(x_{k+1}, y_{k+1}) &= y'(x_{k+1})= y'(x_{k}+h) \\
f(x_{k+1}, y_{k+1}) &= y'(x_k) + hy''(x_k) + \frac{h^2}{2}y'''(\zeta_2)
\end{align*}
Combining these gives
\begin{align*}
T_k &= \frac{h y'(x_k) + \frac{h^2}{2}y''(x_k) + \frac{h^3}{6}y'''(\zeta_1)}{h} - \frac{1}{2}(f(x_k, y(x_k)) + y'(x_k) + hy''(x_k) + \frac{h^2}{2}y'''(\zeta_2)) \\
T_k &= y'(x_k) + \frac{h}{2}y''(x_k) + \frac{h^2}{6}y'''(\zeta_1) - y'(x_k) - \frac{h}{2}y''(x_k) - \frac{h^2}{4}y'''(\zeta_2) \\
T_k &= \frac{h^2}{6}y'''(\zeta_1) - \frac{h^2}{4}y'''(\zeta_2)
\end{align*}
Thus the truncation error of the trapezoid rule is order 2.
\\





\end{document} % This is the end of the document


